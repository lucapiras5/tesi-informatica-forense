% Options for packages loaded elsewhere
\PassOptionsToPackage{unicode}{hyperref}
\PassOptionsToPackage{hyphens}{url}
%
\documentclass[
  12pt,
  a4paper,
]{book}
\usepackage{amsmath,amssymb}
\usepackage{setspace}
\usepackage{iftex}
\ifPDFTeX
  \usepackage[T1]{fontenc}
  \usepackage[utf8]{inputenc}
  \usepackage{textcomp} % provide euro and other symbols
\else % if luatex or xetex
  \usepackage{unicode-math} % this also loads fontspec
  \defaultfontfeatures{Scale=MatchLowercase}
  \defaultfontfeatures[\rmfamily]{Ligatures=TeX,Scale=1}
\fi
\usepackage{lmodern}
\ifPDFTeX\else
  % xetex/luatex font selection
  \setmainfont[]{Libertinus Serif}
  \setmonofont[]{Source Code Pro}
\fi
% Use upquote if available, for straight quotes in verbatim environments
\IfFileExists{upquote.sty}{\usepackage{upquote}}{}
\IfFileExists{microtype.sty}{% use microtype if available
  \usepackage[]{microtype}
  \UseMicrotypeSet[protrusion]{basicmath} % disable protrusion for tt fonts
}{}
\usepackage{xcolor}
\usepackage[top=4cm,bottom=4cm,left=3.25cm,right=3.25cm]{geometry}
\setlength{\emergencystretch}{3em} % prevent overfull lines
\providecommand{\tightlist}{%
  \setlength{\itemsep}{0pt}\setlength{\parskip}{0pt}}
\setcounter{secnumdepth}{5}
\newlength{\cslhangindent}
\setlength{\cslhangindent}{1.5em}
\newlength{\csllabelwidth}
\setlength{\csllabelwidth}{3em}
\newlength{\cslentryspacingunit} % times entry-spacing
\setlength{\cslentryspacingunit}{\parskip}
\newenvironment{CSLReferences}[2] % #1 hanging-ident, #2 entry spacing
 {% don't indent paragraphs
  \setlength{\parindent}{0pt}
  % turn on hanging indent if param 1 is 1
  \ifodd #1
  \let\oldpar\par
  \def\par{\hangindent=\cslhangindent\oldpar}
  \fi
  % set entry spacing
  \setlength{\parskip}{#2\cslentryspacingunit}
 }%
 {}
\usepackage{calc}
\newcommand{\CSLBlock}[1]{#1\hfill\break}
\newcommand{\CSLLeftMargin}[1]{\parbox[t]{\csllabelwidth}{#1}}
\newcommand{\CSLRightInline}[1]{\parbox[t]{\linewidth - \csllabelwidth}{#1}\break}
\newcommand{\CSLIndent}[1]{\hspace{\cslhangindent}#1}
\ifLuaTeX
\usepackage[bidi=basic]{babel}
\else
\usepackage[bidi=default]{babel}
\fi
\babelprovide[main,import]{italian}
\ifPDFTeX
\else
\babelfont{rm}[]{Libertinus Serif}
\fi
% get rid of language-specific shorthands (see #6817):
\let\LanguageShortHands\languageshorthands
\def\languageshorthands#1{}
\usepackage{indentfirst}
\pagestyle{plain}
\frenchspacing
\ifLuaTeX
  \usepackage{selnolig}  % disable illegal ligatures
\fi
\IfFileExists{bookmark.sty}{\usepackage{bookmark}}{\usepackage{hyperref}}
\IfFileExists{xurl.sty}{\usepackage{xurl}}{} % add URL line breaks if available
\urlstyle{same}
% Make links footnotes instead of hotlinks:
\DeclareRobustCommand{\href}[2]{#2\footnote{\url{#1}}}
\hypersetup{
  pdftitle={Uso del software libero per un trattamento scientifico della prova digitale nell'informatica forense},
  pdfauthor={Luca Piras},
  pdflang={it},
  hidelinks,
  pdfcreator={LaTeX via pandoc}}

\title{Uso del software libero per un trattamento scientifico della
prova digitale nell'informatica forense}
\author{Luca Piras}
\date{}

\begin{document}
\frontmatter
\maketitle

{
\setcounter{tocdepth}{2}
\tableofcontents
}
\setstretch{1.5}
\mainmatter
\newcommand{\Omissis}{[\dots\unkern]}
\newcommand{\VediRef}[1]{\footnote{V. sez. \ref{#1}.}}
\newcommand{\VediAdEsempioUrl}[1]{\footnote{V., ad es., \url{#1}.}}
\newcommand{\AutoreTitoloAnnoUrl}[4]{{#1}, \emph{#2}, {#3}, \url{#4}}
\newcommand{\VediUrl}[4]{V. \AutoreTitoloAnnoUrl{#1}{#2}{#3}{#4}}
\newcommand{\vediUrl}[4]{v. \AutoreTitoloAnnoUrl{#1}{#2}{#3}{#4}}

\chapter{Software libero per l'informatica
forense}\label{software-libero-per-linformatica-forense}

\section{Introduzione}\label{introduzione}

I capitoli precedenti hanno dimostrato le potenzialità del software
libero per l'informatica forense da un punto di vista puramente teorico,
e come il software libero fornisce la migliore risposta alle esigenze
tecniche e legali dell'informatica forense.

Fortunatamente, questi vantaggi non sono destinati a rimanere lettera
morta, perché esiste software libero che può essere utilizzato per
l'informatica forense. Questo capitolo darà un resoconto non
esaustivo\footnote{Cercare di individuare tutto il software esistente
  per ciascuna branca della \emph{digital forensics} andrebbe fuori
  dall'ambito della trattazione. In ogni caso, il software è in continua
  evoluzione, e le considerazioni svolte potrebbero non valere nel
  futuro. Pertanto, ci si limiterà a svolgere considerazioni generiche.}
del software libero esistente, incluso l'uso di sistemi operativi
interamente liberi.

\begin{itemize}
\tightlist
\item
  migliore qualità dd
\item
  paper su perché usare linux
\end{itemize}

\section{Software libero}\label{software-libero}

\subsection{Acquisizione di supporti
materiali}\label{acquisizione-di-supporti-materiali}

Il primo passo è l'acquisizione dei dati che saranno oggetto di analisi.

Se i dati risiedono su un supporto materiale che può essere collegato ad
un computer\footnote{Ad esempio, un \emph{hard disk} interno o esterno,
  memorie flash USB o SD, supporti ottici\ldots{}} esistono vari
programmi.

Il più semplice è \emph{GNU dd}.\footnote{V. sez. 11.2, ``dd: Convert
  and copy a file'' in Free Software Foundation, \emph{GNU Coreutils},
  2023.} Il vantaggio principale di \emph{dd} è la sua ubiquità. Dato
che è un comando standard sui sistemi operativi GNU/Linux, se un
computer può eseguire una distribuzione Linux, può anche usare
\emph{dd}.\footnote{Fa parte delle \emph{GNU coreutils}, e la sua
  presenza è richiesta dallo standard \emph{Linux Standard Base}.} Lo
svantaggio principale è la sua semplicità. \emph{dd} è un comando
generico, e non offre meccanismi sofisticati di gestione degli errori, o
informazioni diagnostiche dettagliate.\footnote{Di default, \emph{dd} si
  arresta dopo il primo errore di lettura. È possibile usare le opzioni
  \emph{conv=noerror,sync} affinché \emph{dd} continui a seguito di
  errori, e riempia le parti che non è stato possibile leggere con zeri.}

\emph{ddrescue} è un comando specializzato per copiare dati da supporti
che presentano errori di lettura, ed usa un algoritmo creato \emph{ad
hoc} per cercare di copiare quanti più dati possibile, e causare quanti
meno danni possibile al supporto, durante il suo
funzionamento.\footnote{V. sez. 4, ``Algorithm'' in A. D. Diaz,
  \emph{GNU ddrescue Manual}, 2023.} Inoltre, produce anche un
\emph{mapfile}, che contiene informazioni diagnostiche dettagliate sullo
stato di ogni settore letto dal disco.\footnote{V. sez. 8, ``Mapfile
  structure'', in ibidem.}

\subsection{Acquisizione di embedded
devices}\label{acquisizione-di-embedded-devices}

Se i dati sono su \emph{embedded devices}\footnote{Intesi come
  dispositivi per cui non è possibile estrarre il supporto di memoria
  che contiene i dati. Ad es., smartphones, apparecchiature mediche,
  autoveicoli\ldots{}} è necessario controllare le opzioni disponibili,
caso per caso. In generale, si possono seguire due strade.

Se il dispositivo usa un sistema operativo basato su
GNU/Linux,\footnote{Ad esempio, Android.} ed è possibile eseguire
comandi su quel dispositivo\footnote{Ad esempio, sui dispositivi
  Android, per mezzo di Termux, v
  \url{https://wiki.termux.com/wiki/Main_Page}.}, è possibile usare il
dispositivo stesso per calcolare l'\emph{hash} dei dati prima della
copia, e trasferire i dati all'esterno del dispositivo.

Il vantaggio di questa modalità è che si può presupporre che i risultati
siano affidabili, dato che le operazioni di acquisizione sono state
compiute con software libero. Gli svantaggi sono che si possono estrarre
solo file, e non l'intero supporto, e che l'esecuzione dei comandi
modifica lo stato del dispositivo, rendendo questa operazione
intrinsecamente irripetibile.

Se il produttore del dispositivo offre uno strumento \emph{ad hoc} per
l'estrazione di dati o creazione di backup,\footnote{Ad esempio,
  \emph{iTunes} per gli iPhone}, è possibile eseguirlo all'interno di
una macchina virtuale.\footnote{Ad esempio, con \emph{VirtualBox} o
  \emph{QEMU}. È preferibile usare una macchina virtuale in modo da
  tenere traccia dell'ambiente che è stato usato per l'acquisizione dei
  dati.}

Il vantaggio di questa modalità è che usare strumenti ufficiali, anche
se proprietari, è una garanzia che i risultati dell'estrazione siano
affidabili. Lo svantaggio è che la quantità dei dati estraibili potrebbe
essere ridotta.

Nel campo della \emph{mobile forensics} esistono strumenti di estrazione
di terze parti proprietari, ma è preferibile evitarli. Anche se sono in
grado di estrarre più dati, si pone il problema dell'affidabilità
dell'acquisizione.

\section{Network forensics}\label{network-forensics}

Per la \emph{network forensics}, Wireshark (licenza GPL v2)\footnote{V.
  \url{https://www.wireshark.org}.} è un software maturo, in sviluppo da
più di 20 anni, e permette di acquisire ed analizzare il traffico di
rete.

In particolare, può essere usato per catturare le pagine ed i dati
scaricati per mezzo di un \emph{web browser} (Firefox o Chrome). Con
alcuni accorgimenti aggiuntivi, è possibile usare Wireshark per compiere
un'acquisizione forense di pagine web, inclusi i contenuti sul
\emph{cloud}.\footnote{Per impostare Wireshark in modo che possa
  catturare il traffico generato dal \emph{browser}, v. sez. ``Using the
  (Pre)-Master-Secret'' in
  \url{https://web.archive.org/web/20230724183942/https://wiki.wireshark.org/TLS}.
  Successivamente, è utile visitare il sito internet di una testata
  giornalistica prima e dopo dell'acquisizione, in modo da dimostrare il
  momento in cui i dati sono stati catturati. Infine, è utile eseguire
  le operazioni all'interno di una macchina virtuale, e registrare lo
  schermo, in maniera da lasciare traccia delle operazioni svolte.}

Un'altra strada per copiare dati salvati sul cloud è usare \emph{Rclone}
(licenza MIT)\footnote{V. \url{https://rclone.org/}.} che permette di
copiare dati da un grande numero di servizi di \emph{cloud
storage}.\footnote{Ad esempio, Dropbox, Google Drive, OneDrive\ldots{}
  v. \url{https://rclone.org/overview/}.}

\section{Conservazione}\label{conservazione}

Dopo che i dati sono stati acquisiti, è necessario garantire la loro
corretta conservazione.

Programmi come \emph{BorgBackup} (licenza BSD)\footnote{V.
  \url{https://www.borgbackup.org/}.} e \emph{Restic} (licenza
BSD)\footnote{V. \url{https://restic.net/}.} permettono di creare copie
di backup dei dati, di proteggere i backup con la
crittografia,\footnote{In modo da garantire la confidenzialità dei dati,
  anche nel caso di un \emph{data breach}.} e di verificare la loro
integrità.\footnote{In modo che sia possibile verificare che i dati non
  siano variati per \emph{bit rot} o modifiche intenzionali da parte di
  terzi, anche a distanza di tempo.}

I dati possono essere salvati su \emph{filesystem} liberi e
specializzati per l'archiviazione dei file, come \emph{OpenZFS} (licenza
CDDL), che controlla automaticamente l'integrità dei dati.\footnote{Per
  maggiori dettagli, v.
  \url{https://web.archive.org/web/20231030212040/https://openzfs.github.io/openzfs-docs/Basic+Concepts/Checksums.html}.
  Per una dimostrazione pratica di come ZFS rileva e corregge i dati
  danneggiati, v.
  \url{https://web.archive.org/web/20220516050411/https://ubuntu.com/tutorials/testing-the-self-healing-of-zfs-on-ubuntu}.}

La catena di custodia può essere redatta con \emph{Git}. Ogni operazione
viene registrata in un \emph{commit}, che viene firmato digitalmente.

\section{Zanero}\label{zanero}

E. Huebner, S. Zanero\footnote{\emph{Open Source Software for Digital
  Forensics}, Springer Science+Business Media 2010.}, 25 ss. -- macchine
virtuali

E. Huebner, S. Zanero\footnote{Ibidem.}, 45 ss. -- architettura per
processare grandi quantità di dati

E. Huebner, S. Zanero\footnote{Ibidem.}, 101 ss. -- architetture per
raccogliere malware

E. Huebner, S. Zanero\footnote{Ibidem.}, 117 ss. -- script per estrarre
file

E. Huebner, S. Zanero\footnote{Ibidem.}, 69 ss. -- distribuzioni live
per l'analisi forense

\section{Analisi}\label{analisi}

\begin{itemize}
\tightlist
\item
  Autopsy
\item
  Data carving
\item
  Timeline
\end{itemize}

\section{Presentazione}\label{presentazione}

\begin{itemize}
\tightlist
\item
  Redazione delle relazioni
\end{itemize}

\section{Sistema operativo libero}\label{sistema-operativo-libero}

\begin{itemize}
\tightlist
\item
  Tutti i vantaggi del software libero, applicati all'intero sistema
  operativo
\item
  Uso di GNU/Linux per l'informatica forense
\end{itemize}

Anche il sistema operativo è un software (meglio, una collezione di
software), ed esistono sistemi operativi composti interamente, o quasi
interamente\footnote{In alcuni casi, è necessario includere software
  non-libero per far funzionare alcuni componenti hardware, come la
  connessione Wi-Fi, o la scheda video. Ad esempio, il programma per
  installare la distribuzione GNU/Linux Debian tradizionalmente non
  includeva questo tipo di software, perché per motivi ideologici,
  voleva rimanere un sistema composto interamente da software libero.
  Debian permetteva l'installazione di questo software, ma doveva essere
  un processo manuale, in modo che l'utente fosse consapevole che il
  sistema contiene componenti non-liberi. Tuttavia, a partire da Debian
  12 (rilasciata nel 2023), a seguito di una discussione nel progetto,
  il programma per l'installazione è stato modificato, in modo da
  includere anche i componenti non-liberi.
  V. {Autori di Debian Wiki}, \emph{Firmware}, {2023}, \url{https://web.archive.org/web/20230720195706/https://wiki.debian.org/Firmware}.}
di software libero. Dati i vantaggi del software libero è opportuno
valutare la possibilità di usare questi sistemi operativi, e scegliere
sistemi operativi non-liberi solo se strettamente necessario.

\hypertarget{refs}{}
\begin{CSLReferences}{0}{0}
\leavevmode\vadjust pre{\hypertarget{ref-GNUddrescue}{}}%
A. D. Diaz,
\emph{\href{https://web.archive.org/web/20240109210952/https://www.gnu.org/software/ddrescue/manual/ddrescue_manual.html}{GNU
ddrescue Manual}}, 2023

\leavevmode\vadjust pre{\hypertarget{ref-GNUCoreutilsManual}{}}%
Free Software Foundation,
\emph{\href{https://www.gnu.org/software/coreutils/manual/html_node/index.html}{GNU
Coreutils}}, 2023

\leavevmode\vadjust pre{\hypertarget{ref-Zanero2010}{}}%
E. Huebner, S. Zanero, \emph{Open Source Software for Digital
Forensics}, Springer Science+Business Media 2010

\end{CSLReferences}

\backmatter
\end{document}
